
\chapter{Operators and expressions}

%Until now, I've avoided putting (non-atom) expressions into calls.

In most cases in C, where you write literal values, you can write expressions\footnote{More information:
there are some things which need to be known at \emph{compile time} rather than \emph{run time}.
The most obvious cases of this are \emph{case labels} and array sizes in ANSI-C.}.
An expression is a fragment of code which can be ``evaluated'' --- that is, they can be processed to get a value.
For example:
\texttt{3+4} is an expression which gives the value $7$ when evaluated.
In fact, literal values are also expressions (they evaluate to the value as written).

\section{Arithmetic operators}

The operators $+$ and $-$ work as you would expect.
Division is written with $/$ and multiplication as $*$.

The \% operator (known as ``modulo'') returns the remainder of its arguments;
\texttt{a\%b} returns the remainder when \texttt{a} is divided by \texttt{b}.
For example: $7\%3$ is $1$   and $23\%5$ is $3$.


\subsection{Precedence}
When an expression involves more than one type of operator, the issue of precedence comes up.
That is, some operators are evaluated before others --- this is independent of the order the operators are written in.
The rules for $+,-,*,/$ are the same as for normal arithmetic.

You can look up the precedence for operators, but you can also test them.
For example, $+$ and $*$ could be compared like this:
(Enter and run the following)
\begin{codeblock}
#include <stdio.h>

int main(int argc, char** argv) {
    printf("%d %d %d\n", (2 + 3 * 7), ((2 + 3) * 7),
            (2 + (3 * 7))); // Continues from prev line.
    if ((2 + 3 * 7) == ((2 + 3) * 7)) {
        printf("+ is evaluated first\n");    
    } else if ((2 + 3 * 7)==(2 + (3 * 7))) {
        printf("* is evaluated first\n");
    } else {
        printf("Something has gone horribly wrong\n");
    }
    return 0;
}
\end{codeblock} 

\begin{exercise}
Modify the program above to test relative precedence of \texttt{+} and \texttt{\%}.
\end{exercise}


\subsection{Simple Types}
All of C's simple types are numeric in some way.
This includes types which have other purposes (\texttt{char} and \texttt{bool}).

Numeric types can be classified in a number of ways:
\begin{itemize}
 \item Integer vs Floating point --- \texttt{bool\footnote{\texttt{stdbool.h} must be included to use \texttt{bool} (and \texttt{true} and \texttt{false}).}, char, int} are integer types while \texttt{float} and \texttt{double} are floating point types.
Arithmetic operators which act on integers \emph{will evaluate to an integer result} so for example, \texttt{5/2} is \texttt{2}, not \texttt{2.5}.
Do not make this mistake:
\texttt{float f=2/3;}
\texttt{f} will contain $0$ not $0.66$.
This is because storing a value in a variable does not influence the way the expression is evaluated.

If an expression contains a mixture of types, its arguments will be converted to the most general type.
For example:
\texttt{2/3.0} will give a floating point result because one of its arguments is floating point\footnote{Somewhat confusingly, the default floating point type in C is \texttt{double} not \texttt{float}.}.
You could also achieve this with a \emph{cast}: \texttt{2/(float)3} or multiplication \texttt{(1.0*2)/3}.

 \item Signed vs Unsigned --- integer types can be restricted to store only positive values\footnote{and therefore store higher maximum values using the same number of bits because the sign bit can be used for value storage.}.
 For example: \texttt{unsigned int}, \texttt{unsigned char}\footnote{\texttt{int} defaults to \texttt{signed int} but if you care whether your \texttt{char}s are signed or unsigned, it is better to be explicit.}.
 
 \item size --- types can be modified to indicate how much memory they take\footnote{In C, these modifications are not absolutely defined in the standard.}. 
 Types which take more memory can store a larger range of values.
 For example:  \texttt{short int} and \texttt{long int} are smaller and larger than normal \texttt{int}s respectively.
 These can be combined with \texttt{signed} and \texttt{unsigned}.
\end{itemize}
There are other type modifiers but we'll keep this simple and standard\footnote{For example: \texttt{long long int}  and \texttt{quad double} are non-standard types.}.


To output other types with \texttt{printf}, different placeholders are required:
\begin{itemize}
 \item [\%e \%f \%g] --- all of these will output a \texttt{float} \emph{or} \texttt{double} but they format it in different ways.
 \item [\%c] --- prints a char (as a character rather than as a number).
 \item [\%hd] --- prints a \texttt{short int}.
 \item [\%ld] --- prints a \texttt{long int}.
 \item [\%u]  --- prints an \texttt{unsigned int}.
\end{itemize}
Consult documentation for other combinations.

\section{Bit operators}

These operators act on the individual bits in the value.

\begin{itemize}
 \item \lstinline|a&b| --- bitwise \emph{and} of the two values.
 \item \lstinline:a|b: --- bitwise \emph{or} of the two values.
 \item \lstinline|a^b| --- bitwise \emph{exclusive or} of the two values.
 \item \lstinline|a<<b| --- left shift \texttt{a} by \texttt{b} bits.
 \item \lstinline|a>>b| --- right shift \texttt{a} by \texttt{b} bits.
 \item \lstinline|~a| --- one's complement of \texttt{a}.
\end{itemize}

\section{Relational operators}

These return a boolean value \texttt{true} or \texttt{false}.
Remember that in C, \texttt{bool}s are also integer values.
A 0 indicates false, and any non-zero value is interpretted as true.

\lstinline|a < b, a <= b, a > b, a >= b, a != b, a == b|.
Be careful comparing floating point values, since numerical errors can 
creep in\footnote{A better option for \texttt{a==b} is to check if the difference is small 
enough. $|a-b|<tolerance$ $\Rightarrow$ \lstinline!fabs(a-b)<0.0002!}.

\section{Logical operators}
These look like their bitwise counterparts so it is important not to mix them up.
\begin{itemize}
 \item \lstinline!a && b! --- logical \emph{and}. eg: \lstinline!true && true == true!
 \item \lstinline!a || b! --- logical \emph{or}. eg: \lstinline!false || true == true!
 \item \lstinline|!a|   --- logical \emph{not}. eg: \lstinline@!(a || b) == (!a && !b)@  (deMorgan's law)
\end{itemize}

\lstinline!&&! and \lstinline!||! are ``short-circuiting'' operators: the second argument will not be evaluated unless required (ie: in \lstinline!false && EXP! and \lstinline!true || EXP!, \texttt{EXP} will not be evaluated).

\section{Other operators}

The following are also operators in C but we won't talk about them here.

\begin{itemize}
 \item unary \texttt{*} for pointers (as opposed to arithmetic)
 \item \texttt{->} for pointers
 \item \texttt{+ -} for pointers
 \item \texttt{[ ]} for array subscripts
 \item \texttt{ ? : } --- the ternary operator.
 \item \texttt{,} --- the comma operator.
\end{itemize}

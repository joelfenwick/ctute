
\chapter{Strings and String IO}


\todo{Need to have concepts this section assumes:    chapters on arrays and files}

So far, we have output strings (using the \%s placeholder) but apart from \texttt{argv} in main have avoided string variables and string input.
Strings are stored in arrays of char, however in C you can ask how long a string is but not how big an array is.
This is because strings use a special character with the numeric value $0$ (written as \verb!'\0'!) to mark the end of the string.
So \lstinline!strlen("")==0! that is, the length of the empty string is zero.
This also means that \lstinline!("")[0]=='\0'!.
The \emph{null terminator} is not counted as part of the length of the string.


For example:

\begin{lstlisting}
char single[2];
single[0]='A';
single[1]='\0';
\end{lstlisting}

would make \texttt{single}, the same as \verb!"A"!.
Note the use of \verb!'! around characters and \verb!"! to mark strings.

On the other hand,
\begin{lstlisting}
char single[2];
single[0]='A';
single[1]='B';
\end{lstlisting}
would not make \texttt{single} store \verb!"AB"!, rather it is not storing a proper string at all\footnote{It is actually worse than that.
The most we can say, is that we can't tell whether single stores \texttt{"AB"} or not.
This is because we don't know what follows the array in memory.
If there just happened to be a zero byte next then, when the program looked in \texttt{single} it would see a proper string.
}.
Important note: while strings are stored in arrays of char, that does not mean all arrays of char store strings.

Because strings are stored in arrays, it does not make sense to compare them with \texttt{==}.
Instead, we use standard string functions from \lstinline{<string.h>}.
Compile and run the following program:
\begin{lstlisting}
#include <string.h>
#include <stdio.h>

int main(int argc, char** argv) {
    char msg[10];	
    msg[2]='\0';
    msg[0]='h';
    msg[1]='w';
    printf("strlen(%s)==%d\n", msg, strlen(msg));
    printf("msg==\"hw\" -> %d, strcmp(msg,\"hw\")==%d\n", (msg=="hw"), strcmp(msg, "hw"));
    return 0;
}
\end{lstlisting}

In this case, both comparing with \texttt{==} and comparing with the string compare function (\texttt{strcmp}) both give zero/\lstinline!false!.
However, we need to be clear about what \texttt{strcmp} returns.
\begin{lstlisting}
strcmp("B","A")==-1 // arguments are in reverse order
strcmp("B","B")==0  // arguments are the same
strcmp("A","B")==1  // arguments are in correct order
\end{lstlisting}
That is, strcmp returns \emph{0} if strings are the same.

To copy strings, use \texttt{strcpy}:
\lstinline{strcpy(buffer, "hello ");}
String functions which modify a string, put the destination first.
To add to an existing string, use \texttt{strcat} (string concatenate):
\lstinline{strcat(buffer, "world");}
After executing those two statements, \texttt{buffer} will contain \lstinline!"hello world"! \emph{provided that \texttt{buffer} was big enough}.
If buffer was declared as \lstinline!char buffer[12]! or bigger then we are ok.
If it was smaller (eg 2 chars), then the string functions would write off the end of the array and corrupt other variables or crash (possibly)\footnote{This possibly is the worst part.
The program \emph{might} crash or it might look fine.
The behaviour might change depending on which system you ran the program on.
Undefined and unpredictable behaviour like this can be very hard to find and debug.}

What if you want to put numbers into a string? 
The simplest way is with \emph{s}\texttt{printf}:
\begin{lstlisting}
char buffer[20];
sprintf(buffer, "%d + %d == %d", 3,5,3+5);
\end{lstlisting}
Will result in \texttt{buffer} storing \lstinline!"3 + 5 == 8"!.
As with all standard functions which modify strings, \texttt{sprintf} will put a \lstinline!'\0'! on the end of the string it makes.
There is also an \texttt{sscanf} to read values out of a string instead of from a \texttt{FILE*}.

If you are making a string which will be passed outside the current function, it is better to use dynamic arrays:
\begin{lstlisting}
char* getMessage1(void) {
    char* msg=malloc(sizeof(char)*11);
    sprintf(msg, "%d + %d == %d", 3,5,3+5);
    return msg;
}

char* getMessage2(void) {
    char msg[11];
    sprintf(msg, "%d + %d == %d", 3,5,3+5);
    return msg;		// returning pointer to local variable
}
\end{lstlisting}

\texttt{getMessage2} is dangerous because the memory used for array \texttt{msg} be used for other things once the function returns.
Since \texttt{getMessage1} allocates it's memory is on the heap and so won't be reused until it is \texttt{free}d.

How can we read in strings?
You could use (f)scanf like this:
\begin{lstlisting}
char* buffer=malloc(sizeof(char)*5);
fscanf(f, "%s", buffer); // read a string and store it in buffer
\end{lstlisting}
To read the user's response of \texttt{hello}.
But there are some problems here:
\begin{itemize}
 \item The buffer array is big enough to hold the characters h-e-l-l-o, but not big enough to hold the end of string terminator.
 \item So, increase the limit to 6. It will still be unstable if someone enters \texttt{greetings and saluations}.
 \item Ok, make the fixed size buffer big enough to hold anything that could be entered.
 This is (almost always) the wrong way to approach the problem\footnote{The exception is if you know that valid input can not be bigger than a known size.
 Even then, you need to consider what happens if you are fed, \emph{invalid} input which is bigger.}.
 
 For now though, let's change the size to 200 and enter \texttt{greetings and saluations}.
 \item The string in \texttt{buffer} will be "greetings".  
 This is because \texttt{\%s} as an input placeholder only reads up to the next whitespace, not a whole line.
\end{itemize}

So how should we read in strings in view of the fact that string operations do not check that there is enough room for the string?
\emph{Carefully}.
You can read a line of text using \texttt{fgets}:
\begin{lstlisting}
char buffer[81];
char* result=fgets(buffer,81, stdio);
if (!result) {
   // we hit end of file
} else {
   // do something with the string
}
\end{lstlisting}
The \texttt{fgets} function will read (up to) one less than the number you give it to leave room for the \lstinline!'\0'!.
Note that if the line was longer than 80 characters, it won't all be read.

The alternative is to read input a character at a time (possibly storing it in a string which you expand as needed) and then processing it.
\begin{lstlisting}
char* buffer=malloc(sizeof(char)*80);
int c;
do {
    c=fgetc(file);
    if (c!=EOF) {
      // process c
    }
} while (c!=EOF);
\end{lstlisting}
Note that \texttt{fgetc} returns an \texttt{int} not a \texttt{char}.
This is because it returns either a character \emph{or} the special extra value EOF.
Forgetting this leads to bugs.

\section*{String summary}
\begin{itemize}
 \item Strings are stored in \texttt{char} arrays and must be terminated properly.
 \item Standard functions will take care of \lstinline!'\0'! for you. In your own code, you need to look after it.
 \item C won't check that the underlying storage is big enough.
 \item \texttt{strlen(s)} --- length of \texttt{s}.
 \item \texttt{strcpy(d,s)} --- make \texttt{d} a copy of \texttt{s}
 \item \texttt{strcat(d,s)} --- copy \texttt{s} onto the end of \texttt{d}.
 \item \texttt{fgets} and \texttt{fgetc} are your freinds.
\end{itemize}

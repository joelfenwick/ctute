
\chapter{Enums}

An enumerated type is one which has a set of discrete possible values, each with a symbolic name. 
These values are mapped to integer values for storage.

\section{Usage}

For example:
\begin{codeinline}
enum Day {
    SUNDAY,
    MONDAY,
    TUESDAY,
    WEDNESDAY,
    THURSDAY,
    FRIDAY,
    SATURDAY
};
\end{codeinline}

This allows you to write things like:
\begin{codeinline}
enum Day next_day(enum Day d);  // pass in and out of functions

Day d=MONDAY;  // declare variables

while (d!=WEDNESDAY) ...

switch(...) {   
case SUNDAY:...
...
};
\end{codeinline}


You can specify values for enumerated types
\begin{codeinline}
enum Day {
    SUNDAY=1,
    MONDAY=2,
    ...
};
\end{codeinline}

If you don't specify values, the compile will choose them.

\section{Typedef-ing}
As with other types, you can use typedef to avoid having to write \texttt{enum} whenever you use them:
\begin{codeinline}
typedef enum {
...
} Day;

Day next_day(Day d);
\end{codeinline}


\section{What they are really}

As noted above, enums are stored as integers.
Unlike in some other languages, enums are not treated as a separate type by the compiler.
It will not prevent invalid numeric values being stored in enum variables.

For example, the following are legal:
\begin{codeinline}
Day d=SATURDAY;
d++;

typedef enum {
    RED,
    GREEN,
    BLUE
} Colour;
Day d=GREEN;
\end{codeinline}

If the name of an enum value could appear in multiple contexts, it's a good idea to add a prefix to tell them apart.
For example:
\begin{codeinline}
typedef enum {
    PRESTART=0,
    INIT=1
    PREPROCESS=2,
    DONE=3
} InputState;

typedef enum {
    START=0,
    PROCESS=1,
    DONE=2      // Ambiguous with InputState version
} OuputState;
\end{codeinline}

Something like the following would avoid this:
\begin{codeinline}
typedef enum {
    IS_PRESTART=0,
    IS_INIT=1
    IS_PREPROCESS=2,
    IS_DONE=3
} InputState;
\end{codeinline}



